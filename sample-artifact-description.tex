% This is a template for producing artifact descriptions associated with ECOOP 2017 papers
%
% Following is the notice from Camil Demetrescu's ECOOP 2016 template on which this
% template is based:
% This is a template for producing artifact descriptions associated with ECOOP papers
% Adapted from the standard LIPIcs template by Camil Demetrescu
% See lipics-manual.pdf for further information.
% April 22, 2015

\documentclass[a4paper,UKenglish]{darts}
% for A4 paper format use option "a4paper", for US-letter use option "letterpaper"
% for british hyphenation rules use option "UKenglish", for american hyphenation rules use option "USenglish"
% for section-numbered lemmas etc., use "numberwithinsect"
 
\usepackage{microtype}%if unwanted, comment out or use option "draft"

%\graphicspath{{./graphics/}}%helpful if your graphic files are in another directory

\bibliographystyle{plainurl}% the recommended bibstyle

% ARTIFACT: Include the following input command here
\input{artifactcmds}

% Author macros::begin %%%%%%%%%%%%%%%%%%%%%%%%%%%%%%%%%%%%%%%%%%%%%%%%
% ARTIFACT: Please use the same title as the corresponding ECOOP paper and append the text "(Artifact)"
% ARTIFACT: Add as a footnote the reference to the corresponding ECOOP paper
\title{Your ECOOP paper title (Artifact)\footnote{This artifact is a companion of the paper:  John Q. Open and Joan R. Access, ``Your ECOOP paper title'', Proceedings of the 31st European Conference on Object-Oriented Programming (ECOOP 2017), June 18-23, 2017, Barcelona, Spain. This work was supported in part by\ldots}}
\titlerunning{Your ECOOP paper title (Artifact)} %optional, in case that the title is too long; the running title should fit into the top page column

% ARTIFACT: Authors may not be exactly the same as the ECOOP paper, e.g., you may want to include authors who contributed to the preparation of the artifact, but not to the ECOOP companion paper
%% Please provide for each author the \author and \affil macro, even when authors have the same affiliation, i.e. for each author there needs to be the  \author and \affil macros
\author[1]{John Q. Open}
\author[2]{Alice W. Source\footnote{Core artifact developer.}}
\affil[1]{Dummy University Computing Laboratory, Address/City, Country\\
  \texttt{open@dummyuniversity.org}}
\affil[2]{Department of Informatics, Dummy College, Address/City, Country\\
  \texttt{access@dummycollege.org}}
\authorrunning{J.\,Q. Open and A.\,W. Source} %mandatory. First: Use abbreviated first/middle names. Second (only in severe cases): Use first author plus 'et. al.'

\Copyright{John Q. Open and Alice W. Source}%mandatory, please use full first names. LIPIcs license is "CC-BY";  http://creativecommons.org/licenses/by/3.0/

\subjclass{Dummy classification -- please refer to \url{http://www.acm.org/about/class/ccs98-html}}% mandatory: Please choose ACM 1998 classifications from http://www.acm.org/about/class/ccs98-html . E.g., cite as "F.1.1 Models of Computation" -- ARTIFACT: You may use the same as the corresponding ECOOP paper.

\keywords{Dummy keyword -- please provide 1--5 keywords}% mandatory: Please provide 1-5 keywords -- ARTIFACT: You may use the same as the corresponding ECOOP paper.
% Author macros::end %%%%%%%%%%%%%%%%%%%%%%%%%%%%%%%%%%%%%%%%%%%%%%%%%

%Editor-only macros:: begin (do not touch as author)%%%%%%%%%%%%%%%%%%%%%%%%%%%%%%%%%%
\Volume{3}
\Issue{2}
\Article{1}
\RelatedConference{European Conference on Object-Oriented Programming (ECOOP 2017), June 18-23, 2017, Barcelona, Spain}
% Editor-only macros::end %%%%%%%%%%%%%%%%%%%%%%%%%%%%%%%%%%%%%%%%%%%%%%%

\begin{document}

\maketitle

\begin{abstract}
  This artifact is based on {\tt ArtiFact}, a dynamic program analysis tool that
  can profile paths spanning multiple loop iterations in the control flow
  graph of a Java program. Profiled paths are obtained as the concatenation of
  up to $k$ acyclic paths, where $k$ is a user-defined parameter. The profiler was
  implemented as a patch to the Jikes Research Virtual Machine. The provided
  package is designed to support repeatability of the experiments of the
  companion paper: in particular, it allows users to test the profiler on a variety of
  benchmarks and includes detailed instructions and scripts for running them
  and for visualizing and interpreting collected profiles. Instructions for
  rebuilding the profiler from scratch in the Jikes RMV are also provided.
 \end{abstract}

% ARTIFACT: please stick to the structure of 7 sections provided below

% ARTIFACT: section on the scope of the artifact (what claims of the paper are intended to be backed by this artifact?)
\begin{scope}
  The artifact is designed to support repeatability of all the experiments of the 
  companion paper, allowing users to test the profiler on a variety of benchmarks. In particular, \ldots
\end{scope}

% ARTIFACT: section on the contents of the artifact (code, data, etc.)
\begin{content}
  The artifact package includes:
  \begin{itemize}
  \item a Jikes RVM patch that allows users to profile hot methods of Java programs;
  \item a Java program for visualizing the reports generated by the profiler using GraphViz;
  \item detailed instructions for using the artifact and for rebuilding it from scratch, provided as an {\tt index.html} file.
  \end{itemize}
  To simplify repeatabiliy of our experiments, we provide a VirtualBox disk
  image containing a lightweight Linux distribution fully configured for
  testing our profiler. The image contains Precise Puppy 5.6 - a stripped-down
  Linux distribution based on Ubuntu 12.04 LTS (Precise Pangolin). In
  particular, since the process of building the Jikes RVM can be
  time-consuming (several tools, such as the GNU Classpath and JUnit, are
  downloaded and built during the first source compilation), this image comes
  up with an already patched and compiled Jikes RVM build, and includes all
  benchmarks used in the paper. The Jikes RVM has been compiled with OpenJDK
  1.6.0\_27 and gcc 4.6.3 using the default production configuration as
  suggested on the project's website. 
\end{content} 

% ARTIFACT: section containing links to sites holding the
% latest version of the code/data, if any
\begin{getting}
% leave empty if the artifact is only available on the DROPS server.
% otherwise, provide links to the latest version of the artifact (e.g., on github)
  The latest version of our code is available on the Jikes RVM Research Archive:
  {\url http://sourceforge.net/p/jikesrvm/research-archive/41/} and at 
  {\url http://www.cs.fu.utopia/$\sim$authorthree/projects.html}.
\end{getting} 

% ARTIFACT: section specifying the platforms on which the artifact is known to
% work, including requirements beyond the operating system such as large
% amounts of memory or many processor cores
\begin{platforms}
  The artifact is known to work on any platform running Oracle VirtualBox
  version~4 ({\url https://www.virtualbox.org/}) with at least 5~GB of free
  space on disk and at least 2~GB of free space in RAM.
\end{platforms}

% ARTIFACT: section specifying the license under which the artifact is
% made available
\license{EPL-1.0 ({\url http://www.eclipse.org/legal/epl-v10.html})}

% ARTIFACT: section specifying the md5 sum of the artifact master file
% uploaded to the Dagstuhl Research Online Publication Server, enabling 
% downloaders to check that the file is the expected version and suffered 
% no damage during download.
\mdsum{3fda3a372cea21a28d433519a4412ded}

% ARTIFACT: section specifying the size of the artifact master file uploaded
% to the Dagstuhl Research Online Publication Server
\artifactsize{1.2 GB}

\subparagraph*{Acknowledgements}

The authors wish to thank \dots

% ARTIFACT: optional appendix
%\appendix

%\section{My Appendix}

% Add here any further material you would like to include. For instance, if the artifact is itself a PDF document, add it here.


% ARTIFACT: include here any additional references, if needed...

%% Either use bibtex (recommended), but commented out in this sample

%\bibliography{dummybib}

%% .. or use the thebibliography environment explicitely

\nocite{Simpson}

\begin{thebibliography}{50}
\bibitem{Simpson} Homer J. Simpson. \textsl{Mmmmm...donuts}. Evergreen Terrace Printing Co., Springfield, Somewhere, USA, 1998
\end{thebibliography}


\end{document}
